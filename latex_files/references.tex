\phantomsection\addcontentsline{toc}{section}{Références}
\begin{thebibliography}{ABC}	
%%    \bibitem[REF]{reference} auteur. \emph{titre}. édition, année.
%%    \bibitem[LPP]{lpp} Rolland. \emph{LaTeX par la
%%    pratique}. O'Reilly, 1999.


\bibitem{reference} E. Masiello, \emph{Inférence statistique des copules}, ISFA, support de cours, 2015.

\bibitem{reference} A. Charpentier, C. Dutang, \emph{L'actuariat avec R}, 2012.

\bibitem{reference} E. Marceau, \emph{Modélisation et évaluation des risques en actuariat}, Springer, 2013.

\bibitem{reference} C. Kharoubi-Rakotomalala, \emph{Les fonctions copules en finance}, Sorbonensia oeconomica, 2008.

\bibitem{reference} A. Charpentier, \emph{Copules et risques multiples}, Université de Rennes 1.

\bibitem{reference} J. Yan, \emph{Enjoy the Joy of Copulas: With a Package copula}, Journal of Statistical Software, Volume 21, Issue 4, October 2007.

\bibitem{reference} J.-C. Boies, \emph{Une méthode graphique de détection de la dépendance}, Université de Laval,
Québec, 2003.

\bibitem{reference} R. B. Nelsen, \emph{Dependence Modeling with Archimedean Copulas}, 2005.

\bibitem{reference} P. Deheuvels, \emph{La fonction de dépendance empirique et ses propriétés. Un test non paramétrique d'indépendance}, Académie Royale de Belgique.

\bibitem{reference} C. Genest, A.-C. Favre, \emph{Everything You Always Wanted to Know about Copula Modeling but Were Afraid to Ask}, 2007.

\bibitem{reference} K. Armel, F. Planchet, A. Kamega, \emph{Quelle structure de dépendance pour un générateur de scénarios économiques en assurance ? Impact sur le besoin en capital}, Université Lyon 1, 2010.

\bibitem{reference} Régis Lebrun, \emph{Contributions à la modélisation
de la dépendance stochastique}, Université Paris Diderot, 2013. 







\end{thebibliography}

%% \textbf{Sources des données financières sur l'inflation} :

%% \url{www.inflation.eu}
