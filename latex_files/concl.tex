\section*{Conclusion}
\addcontentsline{toc}{section}{Conclusion}

Lors de ce projet, nous avons pu mettre en application la théorie des copules. Cette théorie voit tout son intérêt dans la gestion des risques et plus généralement dans le domaine de l'actuariat et de la finance. Grâce à ce puissant outil statistique, nous avons cherché à proposer une modélisation de la dépendance entre les valeurs de la vitesse maximale du vent enregistrées par deux stations distinctes. Les procédures d'estimation des copules ont été réalisées selon diverses méthodes (paramétriques, semi-paramétriques et non-paramétriques) afin d'estimer au mieux les paramètres de chaque copule (et les paramètres des lois marginales). Ensuite, nous avons retenu la moyenne des paramètres obtenu par ces différentes méthodes pour réaliser un ensemble de tests graphiques et de tests statistiques afin de comparer l'adéquation des modèles de copules étudiés à nos données. Ainsi, il ressort de cette étude que la copule normale permet de représenter au mieux la dépendance positive existant entre les valeurs des deux stations.



