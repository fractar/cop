
\section{Présentation des données}
%%%%%%%%%%%%%%%%%%%%%%%%%%%%%%%%%%%%%%%%%%%%%%%%%%%%%%%%%%%%%%%%%%%%%%%
\subsection{Présentation}

\subsection{Observation graphique}

\subsubsection{Scatter plot}

\subsubsection{Rank-rank plot}

\subsection{Corrélation de Pearson :  pas une mesure d’association !}

\subsection{Rho de Spearman}

\subsection{Tau de Kendall}

\section{Méthode non paramétrique d'estimation : la copule empirique}
%%%%%%%%%%%%%%%%%%%%%%%%%%%%%%%%%%%%%%%%%%%%%%%%%%%%%%%%%%%%%%%%%%%%%%%
\subsection{Choix d'une copule théorique}


\section{Méthode paramétriques d'estimation}
%%%%%%%%%%%%%%%%%%%%%%%%%%%%%%%%%%%%%%%%%%%%%%%%%%%%%%%%%%%%%%%%%%%%%%%

\subsection{Méthode du maximum de vraisemblance}

\subsection{Méthode IFM (Inference Function for Margins)}


\section{Méthode semi-paramétrique d'estimation (CML)}
%%%%%%%%%%%%%%%%%%%%%%%%%%%%%%%%%%%%%%%%%%%%%%%%%%%%%%%%%%%%%%%%%%%%%%%

Dans cette section, nous allons utiliser la méthode CML (méthode semi-paramétrique) afin d'estimer le paramètre des différentes copules retenues pour modéliser la dépendance entre les données sur Saint-Martin et celles sur Echirolles.

\subsection{Copule de Clayton}

\subsection{Copule de Gumbel}

\subsection{Copule de Franck}

\subsection{Copule normale}

\subsection{Copule de Student}

\section{Méthode des moments : inversion du tau de Kendall}
%%%%%%%%%%%%%%%%%%%%%%%%%%%%%%%%%%%%%%%%%%%%%%%%%%%%%%%%%%%%%%%%%%%%%%%

Dans cette partie, nous utiliserons la méthode des moments pour estimer le paramètre des copules retenues afin de modéliser la dépendance entre les données sur Saint-Martin et celles sur Echirolles.

\subsection{Copule de Clayton}

\subsection{Copule de Gumbel}

\subsection{Copule de Franck}

\subsection{Copule normale}

\subsection{Copule de Student}

\section{Test graphique d'adéquation à la copule : le Kendall plot}
%%%%%%%%%%%%%%%%%%%%%%%%%%%%%%%%%%%%%%%%%%%%%%%%%%%%%%%%%%%%%%%%%%%%%%%

\subsection{Dépendogramme empirique et dépendogramme théorique}

\subsection{K-plot}

\section{Test statistique d'adéquation à la copule : le Kendall plot}
%%%%%%%%%%%%%%%%%%%%%%%%%%%%%%%%%%%%%%%%%%%%%%%%%%%%%%%%%%%%%%%%%%%%%%%

\subsection{Test de Kolmogorov-Smirnov}

\subsection{La statistique de Cramér-von Mises}

\subsection{Estimation du seuil critique par bootstrap paramétrique}





TODO   TODO   TODO   TODO



%% \noindent%
%% \begin{figure}[H]
%%     \begin{center}
%%       \includegraphics[width=17 cm, angle=0]{./pictures/logchrono1.png}
%%       \centering\caption{\label{2}Logarithme du total des importations de gaz naturel}
%%     \end{center}
%% \end{figure}



